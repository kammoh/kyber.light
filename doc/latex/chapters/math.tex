\chapter{Mathematical Background}

A set $F$, which can be initiatively considered as collection of elements, is a fundamental abstraction in most branches of mathematics.
Set $F$ is finite if it contains a finite number of elements.

The algebraic structure $(F, +)$ formed by set $F$ and binary operation + over F is called an Abelian Group, if F is closed, 
associative, and commutative over +, and includes a zero element. In addition every element has an additive inverse 
$-a$ such that $a + -a = 0$. 

$(F, +, .)$ is a Ring if $(F, +)$ forms an Abelian group and $(F, .)$ is closed and . is distributive over +
Commutative Ring
Integral Domain 


A field (a.k.a Galois Field) is a Ring where every non-zero element has a multiplicative inverse. A finite field is a field with
finite number of elements. 

A polynomial in the commutative ring R is $f(x) = a_n x^n + a_{n-1}x{n-1} \dots a_1 x^1 + a_0$ where $a_i \in R$ and $n \geq 0$.

Each $a_i$ is called a \emph{coefficient}
\emph{Polynomial ring} $R[x]$ is the ring formed by polynomials with coefficients in $R$ and operations addition and multiplication.

$g(x) = q(x) f(x) + r(x) $ $deg r(x) < deg f(x)$

$f(x)$ with positive degree is an irreducible polynomial if $\notexists g(x)h(x) = f(x)$
$ GF(p^m) = \mathbf{Z}_p/f(x)$ 
addition and multiplication:






