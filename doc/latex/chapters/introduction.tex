% !TeX root = ./main.tex

\setchapterpreamble[uc][\textwidth]{%
\dictum[Chirrut Îmwe, \textit{Star Wars}]{%
\hfill ``The strongest stars have hearts of \emph{Kyber}''}\vskip1.5em}

\chapter{Introduction and Motivation}

Many aspects of our modern life depend on the reliable functionality of cryptographic algorithms. Public Key (assymentric key) cryptography
 schemes occupy a central role in modern cryptographic applications and with the
 proliferation of Internet of Things (IoT) devices, their presence is expected to become even more pronounced. \\
% The change to front

The security of crypto relies on intractability of certain problems to 
modern computers \cite{moody2017}.

The idea of Quantum Computers has been around for a long time, but it's only in recent years that 
a broad consensus has emerged on their viability.
Bold investments from technology giants such as Google, IBM, Microsoft, and Intel as well as  
 government initiatives herald a not-too-distant future where a quantum computer is able to solve problems
 which are infeasible [to solve] using digital computers. While this immense power is expected to open new horizons in 
 many fields, it will also pose a grave threat to the security of currently deployed Public Key schemes. 
 As formulated by Shor’s algorithm, an adequately powerful quantum computer would be able to solve the underlying hard problems
 used in public key schemes, such as ``integer factorization''
   and ``discrete logarithm'', in polynomial time.


Cryptographic schemes resistant against quantum computing attacks are known as Post-Quantum Cryptography (PQC)
 and are being actively studied and evaluated worldwide. In the US, the National Institute of Standards and 
 Technology (NIST) has organized an evaluation project towards standardizing Post-Quantum
  Cryptography schemes. During the first round, 23 signature schemes and 59
   encryption/KEM (Key Encapsulation) schemes were submitted, of which 69 total were deemed complete and
    proper. 
    
    
  Among the encryption and KEM schemes, those based on Lattices seem the most 
    promising as they're known to provide a high level of security, with acceptable performance
     and practical key size. 

    Lattice-based schemes are more versatile than code-based, MQ, and hash-based schemes [cite poppleman 2014]


    While there has been large amount of research on theory and security foundations of lattice schemes,
     implementation aspects of these schemes have only recently been thoroughly evaluated.





- PQC run on classical computers.
- Unlike RSA and ECC there's even no sub-exponential attack on Lattice-based schemes.
- String security proofs, based on "worst-case" hard problems:..




Lattice problems
trapdoor function
----------
functions that are easy to compute in one way but very difficult to compute in the inverse order





Kyber.PKE\cite{bos2018} is a Public Key Encryption scheme, developed as a part of Crystal Kyber KEM, 
and is one of the leading candidates among NIST Round 1 submissions in the PKE and KEM category.
 It’s based on Module Learning With Error (M-LWR) problem and by itself is deemed to have
  the \textit{indistinguishability under chosen-plaintext attack} (IND-CPA) property. The Public
   Key Encryption scheme is composed of three basic primitives: Key generation (Gen), 
   Encryption (Enc), and Decryption (Dec). The Kyber.PKE submission to NIST includes full
    specifications and a reference C implementation 
    \cite{kyberres}. High-level data-flow diagrams of
     the encryption and decryption primitives are provided in Fig \ref{fig:kyper_enc} and \ref{fig:kyper_dec}.

    %  \begin{table}
    %     \begin{adjustbox}{width=1.2\textwidth,center=\textwidth}
    %         \rowcolors{2}{lightgray!45}{white}   
    %         \begin{tabular}{l*{9}{c}}
    %           \rowcolor{DispositionColor!50}              % Heading with different color to highlight it      
    %                             & n   & k  & q    & $\eta$ & $(d_u,d_v,d_t)$ & $\delta$ & sec & $|pk|$ in bytes & $|c|$ in bytes \\ 
    %           \hline
    %           \textbf{Paranoid} & 255 &  4 & 7681 & 3      & (11, 3, 11)  & $2^{-169}$ & 218 & 1440  & 1536 \\
    %           \textbf{Light}    & 255 &  2 & 7681 & 5      & (11, 3, 11)  & $2^{-145}$ & 102 & 736  & 832 \\
    %          \hline
    %         \end{tabular}
    %        \end{adjustbox}
    %     \caption{Kyber Parameter Sets \cite{bos2018} }
    %     \label{table:params}
    %     \end{table}

% \begin{figure}[h]
%  \includegraphics[width=\textwidth]{../kyber_enc_hl.pdf}
%  \caption{High-level diagram of Kyber.PKE Encryption}
%  \label{fig:kyper_enc}
% \end{figure}

% \begin{figure}[h]
%  \includegraphics[width=\textwidth]{../kyber_dec_hl.pdf} 
%  \caption{High-level diagram of Kyber.PKE Decryption}
%  \label{fig:kyper_dec}
% \end{figure}  

This project aims to design and develop an FPGA implementation of Kyper.PKE optimized 
for area and resource usage. Resilience of the implementation agains timing and power
 attacks is also a primary design consideration. At the time of this writing, no other
  published hardware implementations of Kyber seems yet to be available.

  \chapter{Design Entry}
  The design will be implemented in VHDL 2008 hardware description language. Kyber.PKE 
  will be implemented as distinct entities: Encryption, Decryption, and Key Generation.
   The design will follow best coding practices for code readability and will target a 
   modular design with as much code reuse as possible. Every building block will have 
   it’s own verification test-bench (through CocoTB) and the completed 3 entities will 
   as well be verified against the reference implementation.


The primary target platform is Xilinx Zynq-7000 FPGA SoCs. The developed design will be
 tested on a MYIR Z-turn development board which hosts a Zynq-7020 chip. Developed design
  will be programmed into Zynq’s Programmable Logic (PL) FPGA fabric and accessed through 
  an AMBA-AXI4 interface by Linux application running on the ARM processors (PS). 
  Simulations for verification will be primarily done using GHDL free/open-source VHDL
   simulator.  FPGA synthesis is performed using Vivado. The Integrated Design Environment
    of choice for the VHDL development is Sigasi Studio.


\chapter{Verification}
Ensuring the correct functionality of the implementation is a primary concern during the 
development. The design will be developed in a modular fashion accompanying unit-tests 
and integration tests. Functional verification is performed against the reference 
software implementation using Python, Cocotb \cite{cocotb}
 , and GHDL \cite{ghdl}.

 \chapter{Use of other code}

 Bits and pieces from PoC library \cite{poc} as well as other available open-source libraries could be utilized for the design. 


A survey of hardware implementation of the required building blocks is under way. Ideas and code will be borrowed from previously published implementations and improved upon 
 for the purpose of this project.


 \chapter{Assumptions}

 The design goals in descending priority:
 \begin{itemize}
     \item Area/Resources: measured as the number of used LUT, FF, DSP, and BRAM blocks used as reported by the synthesis tool.
     \item Resilience to Timing Attacks: constant timing
     \item Resilience to Simple Power Attacks
     \item Efficiency: least number of cycles and highest attainable frequency considering all above constraints.  
 \end{itemize}

 \chapter{Circuit Interface}
 The interface of the three main entities (Encryption, Decryption, Key Generation) will 
 be AMBA AXI-Lite which will be accessed both during simulations by CocoTB+GHDL/VCS and 
 in the FPGA instance by the Linux application running on ARM cores. 

 \chapter{Project Timing}

 \begin{table}
 \begin{adjustbox}{width=1.2\textwidth,center=\textwidth}
 \rowcolors{2}{lightgray!45}{white}   
 \begin{tabular}{|c |l |c|}
   \rowcolor{DispositionColor!50}              % Heading with different color to highlight it      
   \textbf{Deadline}  & \textbf{Target} & \textbf{Progress} \\ \hline
   \textbf{Oct 16-17} & Literature, Logistics & 60\% \\
   \textbf{Oct 30-31} & Polynomial Arithmetic, Keccak, Verification Suite &  \\
   \textbf{Nov 13-14} & Gaussian Noise, Integration &  \\
   \textbf{Nov 27-28} & AXI wrapper, FPGA target, Linux, Board verification  &  \\
   \hline
 \end{tabular}
\end{adjustbox}
\caption{Project Timing}
\label{table:schedule}
\end{table}